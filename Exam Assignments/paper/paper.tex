\documentclass[sigconf]{acmart}
\usepackage{booktabs}


%% \BibTeX command to typeset BibTeX logo in the docs
\AtBeginDocument{%
  \providecommand\BibTeX{{%
    \normalfont B\kern-0.5em{\scshape i\kern-0.25em b}\kern-0.8em\TeX}}}

%% These commands are for a PROCEEDINGS abstract or paper.
\settopmatter{printacmref=false} % Removes citation information below abstract
\renewcommand\footnotetextcopyrightpermission[1]{} % removes footnote with conference information in 

% \acmConference[AEPRO 2025]{AEPRO 2025: Algorithm Engineering Projects}{March 1}{Jena, Germany}

% convert text to title case
% http://individed.com/code/to-title-case/

% that helps you to formulate your sentences
% https://www.deepl.com/translator

\begin{document}

%%
%% The "title" command has an optional parameter,
%% allowing the author to define a "short title" to be used in page headers.
\title[Fast Einsum Implementation]{Fast Einsum Implementation Using Parallelization, Vectorization, and More\\\large Algorithm Engineering 2025 Project Paper}

%%
%% The "author" command and its associated commands are used to define
%% the authors and their affiliations.

\author{Jonas Peters}
\affiliation{%
  \institution{Friedrich Schiller University Jena}
  \country{Germany}}
\email{jonas.peters@uni-jena.de}

% \author{Erika Mustermann}
% \affiliation{%
%   \institution{Friedrich Schiller University Jena}
%   \country{Germany}}
% \email{erika.mustermann@uni-jena.de}

%% The abstract is a short summary of the work to be presented in the article.
\begin{abstract}

% The five-finger pattern \cite{macgilchrist2014}:
\begin{enumerate}
% \item \textbf{Topic and background:} What topic does the paper deal with? What is the point of departure for your research? Why are you studying this now?
\item \textbf{Topic and background:} The topic of this paper is tensor transformations, and how to perform them efficiently. Tensors consist of many dimensions. Take matrices as specialized tensors with two dimensions (rows and columns) for example. When multiplying two matrices, we calculate "row times column" for each position in the resulting matrix. In tensor language, this is equivalent to \em contracting \em the the second dimension of the first matrix with the first dimension of the second matrix. We can also just rearrange the dimensions in a new order, which is called  \em transposing\em , similar to matrices. Lastly, we can \em reduce \em entire dimensions by summing over all entries, for example reducing the second dimension of a matrix $M$ is equivalent to
\[ M_{reduced} = M \cdot \vec{1} \quad . \]

In order to not always hardcode these operations, \em einsum \em can be employed by passing in the two operand matrices, together with a format string, which defines how every dimension will be treated. For example \texttt{ik,kj->ij} specifies a normal matrix-matrix-multiplication. Here is how these \em column identifiers \em of \texttt{lhs\_string} in the eisnum expression \texttt{<lhs\_string>,<rhs\_string>-><target\_string>} are interpreted:

\begin{table}[h!]
  \centering
  \caption{Column Identifiers of Einsum Expression}
  \label{tab:column_identifiers}
  \begin{tabular}{lccc}
    \toprule
    \textbf{Category} & \textbf{rhs\_string} & \textbf{target\_string} \\
    \midrule
    Batch              & Yes    & Yes    \\
    Kept left          & No     & Yes    \\
    Contracted         & Yes    & No     \\
    Summed left        & No     & No     \\
    \bottomrule
  \end{tabular}
\end{table}

You have to read the table in the following manner. Take a column identifier of \texttt{lhs\_string}. Then you check whether it is present in \texttt{rhs\_string} and \texttt{target\_string}, and based on these results you assign it to one of the four specified categories, where "Summed left" describes a dimension to reduce.

A similar table can be produced for column identifiers in \texttt{rhs\_string}. The column identifiers in \texttt{target\_string} also specify the resulting shape of the operation. All of the necessary computations to achieve the desired transformation is performed by the einsum function.

\item \textbf{Focus:} What is your research question? What are you studying precisely?
\item \textbf{Method:} What did you do?
\item \textbf{Key findings:} What did you discover?
\item \textbf{Conclusions or implications:} What do these findings mean? What broader issues do they speak to?
\end{enumerate}


\end{abstract}

%%
%% Keywords. The author(s) should pick words that accurately describe
%% the work being presented. Separate the keywords with commas.
\keywords{entity resolution, data cleansing, programming contest}


%%
%% This command processes the author and affiliation and title
%% information and builds the first part of the formatted document.
\maketitle

\let\thefootnote\relax\footnotetext{AEPRO 2025, March 1, Jena, Germany. Copyright \copyright 2025 for this paper by its authors. Use permitted under Creative Commons License Attribution 4.0 International (CC BY 4.0).}


\section{Introduction}


\subsection{Background}

\subsection{Related Work}
I used this~\cite{hptt2017} C++ library 

\subsection{Our Contributions}

\subsection{Outline}



\section{The Algorithm}

\subsection{Internal Representation of Mock Labels}


\subsection{Efficient Preprocessing of Input Data}
\label{sub:sec:preprocessing}

The following findings are important to speed up preprocessing of the input data:

\begin{itemize}
\item Reading many small files concurrently, with multiple threads (compared to a single thread), takes advantage of the internal parallelism of SSDs and thus leads to higher throughput \cite{Zhuang2016}.

\item C-string manipulation functions are often significantly faster than their C++ pendants. For example, locating substrings with \texttt{strstr} is around five times faster than using the C++ \texttt{std::string} function \texttt{find}.

\item Hardcoding regular expressions with \emph{while, for, switch} or \emph{if-else} statements results in faster execution times than using standard RegEx libraries, where regular expressions are compiled at runtime into state machines.

\item Changing strings in place, instead of treating them as immutable objects, eliminates allocation and copying overhead.

\end{itemize}


\section{Experiments}

Table~\ref{tab:results} shows the running times of the resolution step of the five best placed teams.


\begin{table}[htbp]
  \caption{Comparison of the F-measure and the running times of the resolution step of the five best placed teams. The input data for the resolution step consisted of 29{,}787 in JSON formatted e-commerce websites. Measurements were taken on a
laptop running Ubuntu 19.04 with 16 GB of RAM and two Intel Core i5-4310U CPUs. The underlying SSD was a 500\,GB 860 EVO mSATA. We cleared the page cache, dentries, and inodes before each run to avoid reading the input data from RAM instead of the SSD.}
  \label{tab:results}
\resizebox{\columnwidth}{!}{
  \begin{tabular}{lcrr}
    \toprule
    Team& Language & F-measure & Running time (s)\\
    \midrule
	PictureMe (\textbf{this paper}) &C++& 0.99 & \textbf{0.61}\\
    DBGroup@UniMoRe &Python& 0.99 & 10.65\\
    DBGroup@SUSTech &C++& 0.99 & 22.13\\
    eats\_shoots\_and\_leaves &Python& 0.99 & 28.66\\
    DBTHU &Python& 0.99& 63.21\\
  \bottomrule
\end{tabular}
}
\end{table}


\section{Conclusions}

%%
%% The next two lines define the bibliography style to be used, and
%% the bibliography file.
\bibliographystyle{ACM-Reference-Format}
\bibliography{literature}


\end{document}
\endinput
%%
%% End of file `sample-sigconf.tex'.
